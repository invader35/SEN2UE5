\documentclass[a4paper,oneside,openany]{tufte-book}
\usepackage[utf8x]{inputenc}
\usepackage[german]{babel}
\usepackage{microtype} % Improves character and word spacing
\usepackage{booktabs} % Better horizontal rules in tables
\usepackage{ucs}
\usepackage{amsmath}
\usepackage{amsfonts}
\usepackage{amssymb}
\usepackage{graphicx}
\usepackage{color}
\usepackage{listings}
\usepackage{caption}

\makeatletter% since we're using commands with @ in their name

\let\origappendix\appendix% save a copy of the original meaning of \appendix
\renewcommand{\appendix}{%
  \origappendix% do all the original \appendix stuff
  \titlecontents{chapter}%
    [0em] % distance from left margin
    {\vspace{1.5\baselineskip}\begin{fullwidth}\LARGE\rmfamily\itshape} % above (global formatting of entry)
    {\hspace*{0em}\appendixname~\thecontentslabel: } % before w/label (label = ``2'')
    {\hspace*{0em}} % before w/o label
    {\rmfamily\upshape\qquad\thecontentspage} % filler + page (leaders and page num)
    [\end{fullwidth}] % after
  \titleformat{\chapter}%
    [display]% shape
    {\relax\ifthenelse{\NOT\boolean{@tufte@symmetric}}{\begin{fullwidth}}{}}% format applied to label+text
    {\itshape\huge Anhang~\thechapter}% label
    {0pt}% horizontal separation between label and title body
    {\huge\rmfamily\itshape}% before the title body
    [\ifthenelse{\NOT\boolean{@tufte@symmetric}}{\end{fullwidth}}{}]% after the title body
  \setcounter{secnumdepth}{0}% ``number'' the appendices
  \renewcommand{\thefigure}{\@arabic\c@figure}% define \thefigure to use only the figure number (1), not A.1
  \renewcommand{\thetable}{\@arabic\c@table}%
  %
  % Add any other special appendix-related code here.
  %
}
\makeatother% restore the special meaning of @


\author{Schett Matthias}
\title{SEN-\"{U}bung 05}
\begin{document}
\setcounter{tocdepth}{1} 
\definecolor{gray}{rgb}{0.9,0.9,0.9}
\lstset{language=[Visual]Basic, morekeywords={param, local}, breaklines=true, tabsize=4, showstringspaces=false,backgroundcolor=\color{gray}, numbers=left}

\frontmatter

\maketitle
\tableofcontents
\mainmatter

\chapter{Aufgabe 1}

\section{L\"{o}sungsidee}

Um ein Wertepaar abzuspeichern wurde eine Struktur namens DiveData erstellt. Diese Struktur wird dann in einem std::vector abgespeichert.

\begin{lstlisting}[language=C++]
struct DiveData{
    time_t mTime;
    double mDepth;
    double mUpDown;
};

extern std::vector<DiveData> diveComputer;
\end{lstlisting}
\marginnote{Das extern vor der Deklaration des Vektors ist notwendig um Mehrfach Definitionen zu verhindern}

Zum Abspeichern des Eingabestromes wird der scanner Klasse ein Eingabestrom übergeben und dieser anschließend durchgegangen und gelesen, sollten
die Werte nicht den Erwartungen entsprechen wird eine std::exception mit der Meldung <Unknown format> ausgelöst um anzuzeigen, dass ein Fehler aufgetreten ist.

Die Ausgabe erfolgt anschließend in Tabellenform, dafür wurden die Folgenden Manipulatoren erstellt.

\begin{lstlisting}[language=C++]
ostream& hr(ostream& os) {
    return os << "-------------------------
    ------------------------";
}

ostream& colSpace(ostream& os) {
    return os << setw(colSpacing) << " ";
}

ostream& colWidth(ostream& os){
    return os << setw(colWidthNum) << " ";
}

ostream& colFormat(ostream& os) {
    return os << setw(colWidthNum);
}

ostream& formatUpDown(ostream &os){
    return os << right << setiosflags(ios::fixed) << setprecision(4);
}

ostream& formatDepth(ostream &os){
    return os << right << setiosflags(ios::fixed) << setprecision(2);
}
\end{lstlisting}

Der Programmcode befindet sich ab \nameref{chap:Auf1}
\newpage
\section{Testf\"{a}lle}

\lstinputlisting[caption=Input Testfall 1, breaklines=true, tabsize=4, showstringspaces=false,backgroundcolor=\color{gray}]{DiveComputer/Test.txt}
\lstinputlisting[caption=Input Testfall 2, breaklines=true, tabsize=4, showstringspaces=false,backgroundcolor=\color{gray}]{DiveComputer/TestIncorrect.txt}
\lstinputlisting[caption=Ausgabe, breaklines=true, tabsize=4, showstringspaces=false,backgroundcolor=\color{gray}]{DiveComputer/OutputA1.txt}


\chapter{Aufgabe 2}

\section{L\"{o}sungsidee}

Da eine Lagerverwaltung aus einem Lager und das Lager wiederum aus Artikeln besteht, benötigen wir 2 Klassen.
\begin{itemize}
  \item Article
  \item WareHouse
\end{itemize}
Diese beiden Klassen ermöglichen uns dann die Lagerverwaltung.

Ein Artikel besteht aus :
\begin{lstlisting}
    int mArticleNumber;
    std::string mArticleName;
    size_t mQuantity;
    double mPrice;
\end{lstlisting}
Jeder dieser Member bestitzt eine Getter und eine Setter Methode die es erlaubt die Werte zu lesen und zu verändern. Weiters wurde operator< überschrieben um die Lagermenge\sidenote{mQuantity} zu vergleichen.
\begin{lstlisting}
bool Article::operator<(Article const & vgl) const{
    return (mQuantity < vgl.mQuantity);
}
\end{lstlisting}

Das Lager besteht aus:
\begin{lstlisting}
    std::vector<Article> mArticles;
    std::string mWareHouseName;
\end{lstlisting}
Hier bestizt allerdings keiner der Member eine Getter oder Setter Funktion. Es gibt nur eine GetNumberOfArticles Funktion die, die aktuelle Größe des std::vector ausgibt.
Um einen neuen Artikel abzuspeichern gibt es die AddArticle(Article const \&newArticle) Funtkion, die std::vector::push\_back aufruft.
Es gibt auch die Möglichkeit aus einem File einzulesen, dabei hilft abermals die scanner Klasse und liest das File ein, tritt bei der Verarbeitung ein Fehler auf, 
wenn das Format nicht richtig ist, wird eine std::exception geworfen mit der Meldung <Unknown format>.
Die Ausgabe erfolgt anschließend im vorgegebenen Tabellenformat.

Der Quellcode findet sich ab \nameref{chap:Auf2}

\section{Testf\"{a}lle}
\lstinputlisting[caption=Input Testfall 1, breaklines=true, tabsize=4, showstringspaces=false,backgroundcolor=\color{gray}]{WarehouseManager/Input.txt}
\lstinputlisting[caption=Input Testfall 2, breaklines=true, tabsize=4, showstringspaces=false,backgroundcolor=\color{gray}]{WarehouseManager/InputWrong.txt}
\lstinputlisting[caption=Ausgabe, breaklines=true, tabsize=4, showstringspaces=false,backgroundcolor=\color{gray}]{WarehouseManager/OutputA2.txt}

\backmatter

\appendix

\setboolean{@mainmatter}{true}
\chapter{Aufgabe 1}\label{chap:Auf1}
\lstinputlisting[caption=Header für den Tauchcomputer, breaklines=true, tabsize=4, showstringspaces=false,backgroundcolor=\color{gray}]{DiveComputer/DiveComputer.h}
\lstinputlisting[caption=Implementierung des Tauchcomputers, breaklines=true, tabsize=4, showstringspaces=false,backgroundcolor=\color{gray}]{DiveComputer/DiveComputer.cpp}
\lstinputlisting[caption=Testtreiber, breaklines=true, tabsize=4, showstringspaces=false,backgroundcolor=\color{gray}]{DiveComputer/Main.cpp}

\chapter{Aufgabe 2}\label{chap:Auf2}
\lstinputlisting[caption=Header für den Artikel, breaklines=true, tabsize=4, showstringspaces=false,backgroundcolor=\color{gray}]{WarehouseManager/Article.h}
\lstinputlisting[caption=Implementierung des Artikels, breaklines=true, tabsize=4, showstringspaces=false,backgroundcolor=\color{gray}]{WarehouseManager/Article.cpp}
\lstinputlisting[caption=Header für das Lager, breaklines=true, tabsize=4, showstringspaces=false,backgroundcolor=\color{gray}]{WarehouseManager/WareHouse.h}
\lstinputlisting[caption=Implementierung des Lagers, breaklines=true, tabsize=4, showstringspaces=false,backgroundcolor=\color{gray}]{WarehouseManager/WareHouse.cpp}
\lstinputlisting[caption=Testtreiber, breaklines=true, tabsize=4, showstringspaces=false,backgroundcolor=\color{gray}]{WarehouseManager/Main.cpp}

\end{document}